Почему я выбрал данную специальность? Я всегда хотел связать свою жизнь с информационными технологиями, и я очень рад, что мне удалось 
сделать первый шаг.

В старших классах, когда нужно было срочно решать, куда поступать, я никак не мог определиться, перебирал возможные места учебы. По натуре я человек творческий, всегда хорошо рисовал, поэтому и выбрал профессию строителя. В то же время я не типичный гуманитарий: не витаю в облаках, точное мышление и логика у меня также хорошо развиты. На втором курсе я стал понимать, что строительство и архитектура --- это совсем не мое. Многие друзья и знакомые уже работали в сфере IT. Я видел, чем они занимаются, и это было мне намного ближе. В конце концов, зачем выполнять рутинную и повторяющуюся работу, когда компьютер может делать эту работу за вас? Программисты написали множество простых в использовании инструментов, которые облегчают жизнь для них.

Перед тем, как остановить свой выбор на информационных технологиях я читал много статей о том, что меня может ждать. Увидев приставку IT ко множеству разных дисциплин, я решил, что это именно то, что мне нужно. Информационные технологии - это перспективная, бурно развивающаяся отрасль, и с каждым днём она всё больше и больше охватывает все сферы нашей жизни. Для меня одним из преимуществ этой специальности было большое количество практики. Теория - это, конечно, хорошее подспорье, но решение практических задач – все же важнее.

Я также люблю играть в разные компьютерные игры. Меня всегда интересует, кто их создатели, как их создавали, какая работа этому предшествовала и сколько это заняло времени. Еще я часто смотрю фильмы или клипы в которых используется компьютерная графика. Именно она помогает сотворить некий магический эффект, который является главным акцентом. Например, если это фантастические герои, то с помощью компьютерной графики можно воплотить самые невероятные идеи и образы. Поэтому в будущем я хочу работать не только программистом, но и быть создателем интересных компьютерных игр для детей и взрослых.

Я выбрал эту специальность, поставив себе цель, что за эти четыре года я получу как можно больше знаний, приму участие в интересных проектах, займусь научной деятельностью и обязательно найду свой путь в IT-сфере, которая мне так интересна.