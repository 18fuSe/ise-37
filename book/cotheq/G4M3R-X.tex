Все  зародилось в школьные годы, родители купили компьютер, который уж очень сильно меня заинтересовал. У меня был знакомый на то время студент-программист, который показал много интересного, после чего и я сам начал использовать доступные программы и лазить в файлах игр, приложений  и прочего изменять их и смотреть на получаемый результат, хоть и почти совсем ничего не понимал в этом. В самой школе все было не так гладко: во первых <<древние>> компьютерные ну очень слабые по характеристике и с шариковыми мышками…, во вторых молодой учитель информатики, который по итогу вообще ничему не учил, хотя и пытался. Помню, что были попытки обучить Turbo Pascal,  давались простенькие задания, которые без труда мне давались, но продвижения не было.

Я был хорош в учебе и сразу был нацелен на окончание 11 классов. В год окончания учёбы я рассматривал места поступления в ВУЗ, для дальнейшего обучения. Естественно я хотел попасть в тоже место и ту же специальность на которой учился мой ранее упомянутый знакомый. Но проблема заключилась в том, что в ХТИ на привлекательную мне специальность на тот момент, а может быть и сейчас, не было бюджетной формы обучения. Тогда я не мог себе такого позволить, но на случай, если никуда не смогу поступить, подал туда копии документов. Выбора особо и не было, я подал документы в ХГУ на почти такую же специальность, подстраховавшись, также подал копии документов и КПОИиП. Тут надо остановиться поподробней, дело в том, что в колледже на информационную специальность не было набора тех кто закончил 11 классов, только тех кто после 9го класса. Пришлось подать копии документов на не особо привлекательную на тот момент мне специальность тех.обслуживания, на тот момент начинка системного корпуса меня не особо интересовала.  Хоть я и был первым по результатам ЕГЭ в своей школе, в ХГУ по общему конкурсу я не прошел, зато без проблем попал в колледж. Также я прошел по конкурсу и в ХТИ, но по понятным причинам пришлось отказаться.

И вот в 2009 году я стал студентом и был очень заинтересован в обучении, сразу поставил цель продолжить обучение в ХГУ. Но под конец учебного года мне пришлось уйти в академический отпуск, по причине того что надо было <<отдать долг родине!>>, оказалось что тем кто поступил в колледж на базе 11ти классов отсрочка от армии не давалась, что было для меня новостью. Спустя год я без проблем восстановился и продолжил учебу. Скажем так, в 2013 году, успешно закончив учебу, и получив диплом с единственной тройкой. Я прямо наследующий день побежал подавать документы в ХГУ. Спокойно прошел тестирование и поступил. Удачно отучившись год, неожиданно начались проблемы со здоровьем. Далее бег по больницам, от которых по итогу не было толку. Было много пропусков занятий с последствием накоплением долгов, это затянулось аж на год, после чего было принято тяжелое решение забрать документы. После чего пошла череда бесполезных бегов по всяким лекарям и знахарям, а также попытки заработать на перепродаже угля и водителем такси. Руки у меня опустились, стал агрессивным, начал набирать вес, каким-то образом заработал остеохондроз. Помочь мне смогли лишь молодые специалисты, на тот момент я много чего нацеплял и куда они только меня не отправляли и что только не прописывали, но в итоге этого мне помогло, на что я очень сильно им благодарен. После чего я захотел восстановиться и продолжить своё обучение в ХГУ, но на что мне сказали что той специальности, на которой я учился уже нет, и что мне придется опять поступать на 1 курс, и что предметы которые я уже сдавал ранее просто перезачтут. Что я и собирался сделать, но так как  я в 2016 году удачно <<прощелкал>> момент пройти тестирование на поступлении и пришел подавать документы слишком поздно, то пришлось ждать еще год. Я начал готовиться к тестированию, но тут появилась проблема, я стал очень ленив к обучению, что и сильно мешало к подготовке. Лишь в последнии дни перед вступительными испытаниями я все-таки сумел взять себя в руки заставить готовиться к ним, боясь их провалить. Я был очень не уверен в себе в тот момент, но на моё удивление, все прошло гладко, и я даже был среди первых в списке из тех, кто проходил тестирование. Естественно, я вновь смог поступить. Обучение в самом ВУЗе стало в разы лучше, чем это было несколько лет назад, никакого программирования на листочках, работа за нормальными компьютерами. Видимо это и повлияло на мою мотивацию, хоть меня часто охватывает лень, несмотря на это я смог обучиться программированию намного больше всего за один учебный год, нежели за все те года ранее вместе взятые.

Исходя из всей моей исповеди изложенной выше, если сказать вкратце, то еще в школьные годы работа с компьютером и желание создания чего-то своего и повлияло на то, что я пошел в IT-сферу. Также очень сильно повлиял на это мой знакомый, который является для меня примером и по сей день, правда жизнь его занесла и теперь он занимается как раз техническим обслуживанием, к которой почему-то у меня тяги нет.