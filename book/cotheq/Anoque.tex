Компьютер появился в 2009 году, до этого <<шарился>> в ноутбуке брата, когда он приезжал из Красноярска. Интернет появлися через полгода после покупки комьпютера.В то время меня увлекали игры, возможно уже тогда сериалы и документальные фильмы, конечно иногда что-то читал в интернете полезное. Когда-то посмотрел документальный фильм о программистах, хакерах (о них отдельные фильмы). Мне очень понравилось то, что они делали. В принципе я также удивляюсь и каким-либо электронным устройствам. Тогда ещё не думал о том, что хоть как-то буду изучать эту сферу.

Однажды одноклассник попросил меня сделать сайт. Тогда мне было 13 или 14 лет. Конечно я тогда не думал о том, чтобы писать всё самому (даже не представлял как). Самым подходящим вариантом на то момент оказалась CMS под названием DLE (Data life engine). Ну я как и все скачал какой-то шаблон, с очень базовыми знаниями HTML и CSS поправил шаблон под сайт. Функционал конечно не трогал. В целом всё было норм. Только этот школьник оказался ленивым, и ничего не выкладывал на нём. Вот он и загнулся. Даже не помню на каком хостинге покупал домен.

После этого в IT я особо не лез. В школе были базовые задачки на паскале, типа переписать какую-нибудь математическую формулу на паскале и поиграться с входными данными. Иногда лазил по исходникам стандартных проектов, менял какие-то мелочи, конечно особо не понимая что и для чего. Примерно тогда же я достаточно много играл, и делал мини-модификации для игр серии Stalker. У меня был репак, где файлы хранятся не в архиве, а все скрипты находятся просто в своих папках. Я менял эти скрипты и получалась в основном дичь. Но мне это нравилось, и делал мелкие штучки <<по приколу>>. 

Момент, когда начал думать об IT в качестве профессии --- в 9 классе. Тогда был выбор уходить после 9 или учиться до 11. Тогда выбрал в качестве дополнительных экзаменов физику и информатику.
Планировал идти в 10-11 классы на физмат. Не помню почему, но я решил окончательно поступать после 9 класса куда-нибудь. Единственный для меня вариант был Политехнический колледж на специальности <<Информационные системы>>.
На тот момент я знал паскаль, чуть чуть PHP, возможно JavaScript (не помню когда начал на нём что-нибудь писать), ну и HTML и CSS. Ну и всякие плюшки в системе Windows 7, из-за чего родственники просили помочь с компом (через это все проходят).
Ну и в полите сначала учился на 4 и 5. Первый год --- школьная программа. Из IT мало чего изучил. А вот на 2 курсе понеслось. Тогда узнал про базы данных, delphi.
Совершенно без напряжения всё сдавал на 5. Где-то на 3 курсе вспомнил веб технологии немного, написал первый курсач на делфях с базой данных (данные о студентах, наверно около 7 таблицы с несколькими связями).
Тут самостоятельно начал изучать ООП в языке Java по видеоурокам и некоторым статьям. Было тяжело понять, не знаю почему у преподавателей не справшивал.
Пошёл на первую практику в компанию <<Вебтолк>>. Там познакомился с ООП (уже более менее нормально) и такой прекрасной технологией, как менеджер пакетов. В случае с PHP это Composer.
Там было стандартное задание --- сделать парсер сайта \verb|habrahabr.ru| (тогда он ещё так назывался), и записывать в базу данных последние 10 постов (и перезаписывать, т.е. всегда не больше 10 записей).
Это делалось с мини-фреймворком Slim, ORM для работы с БД (названия не помню какия именно, начало было таким
\begin{verbatim}
ORM::for_table('table')->.......)
\end{verbatim}
и парсер (ещё пару дополнений, типа curl, и для вывода объектов в удобном формате).
За всё это время я ещё ничего клёвого не делал (хоть в какой-то нормальной степени клёвости).
Вот на 4 курсе мы начали изучать C\# и 1C. C\# мне нравился, 1С не очень. На парах занимался фигней. Что-то верстал (адаптивный дизайн), изучал jQuery.
Так вот. Можно сказать, что здесь уже реально переломный момент, когда я начал что-то изучать целенаправленно, почти самостоятельно и вроде как более менее получилось.
Меня направили в Томск на WorldSkills по направлению веб-разработки. Я из всех групп вроде как больше всех шарил в вебе (ага, по сути толком ничего не умел). 
За 3 дня до отъезда сообщили об этом. Я вообще не был готов. Хоть Ajax в jQuery успел узнать, и некоторые плюшки в Php. Съездил, было клёво, но конечно ничего не занял (хотя всё равно был вне конкурса).
Зато понял какие задания и какая атмосфера. После приезда начал изучать js, jQuery, php в целом, фреймворк Yii2 и всякие клёвые штуки в CSS (мощная вещь, если уметь пользоваться Для многих вещей и js не нужен, надо быть хитрым и сообразительным :))
Ну и за пару месяцев уже более менее знал эти языки и фреймворки.
По Хакасии занял 1 место. Поехал ещё через месяц в Казань. Там толком ничего не сделал (да и не пытался <<затащить>>, т.к. опыта практически и не было, а там чувак всё это уже давно изучил и был реальный опыт).
К тому же там был косяк с сервером, из-за чего ajax запросы работали криво (точнее чаще всего просто не выполнялись).

Ну и после этого в основном занимаюсь веб-технологиями. Возможно всегда буду заниматься в этой области. Хотя я иногда пробую что-нибудь новое. я уже понимаю, что нельзя останавливаться и лениться.
Всё постоянно совершенствуется, улучшается, появляются куча всего интересного и мощного. И хочется сделать что-то действительно классное.
Возможно участвие в опен-сорс проектах, возможно уникальные сервисы, может даже стартап (хотя я ленивый, чтобы что-то самому создавать). 
Надеюсь, что я буду полезным разработчиком. 
Нельзя стоять на одном месте, на одной технологии. Конечно мой опыт маленький, разнообразия почти нет, но думаю успею много чего изучить.
Инодга мне кажется, что нифига не могу. Хотя вроде как и получается. Бывают дни, когда совсем лень что-либо изучать, как будто мозг не хочет принимать информацию, хотя сериальчики --- пожалуйста.
В некоторые дни на столько прёт, что смотришь на время и удивляешься --- куда это несколько часов так быстро улетели? :D
Сейчас я стараюсь часто писать всякую фигню и заливать на github. Пробую всякое. Конечно там далеко не всё, т.к. начал заливать что-то около месяца назад. (Можете глянуть, там же видно когда я и что пушил).
По сути там безделушки. Но стараюсь писать каждый день. Сейчас главное закончить один проект и уже можно заняться ещё чем-нибудь. 
Человеческий мозг --- странная штука, в один день тебя прёт, в другой ты --- бесполезная биомасса.