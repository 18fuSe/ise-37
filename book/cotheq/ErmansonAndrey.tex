Информационные технологии это приёмы, способы и методы применения средств вычислительной техники при выполнении
функций сбора, хранения, обработки, передачи и использования данных.

Это определение, термину <<информационные технологии>>, является верным, кратким и точным, но оно ничего не значит для парня
13 лет, который толком никогда не сидел за компьютером. Любой ребенок этого возраста любит игры. Я был не исключением, но
ко всему прочему, я обожал их придумывать сам. Придумывать игры меня научил мой старший брат. С малых лет он открыл для
меня необъятный мир собственных фантазий. Придумывание собственных игр не только оказалось веселым занятием, но так же
очень хорошо развивало логическое мышление. Как правило, я придумывал настольные игры, как простые, так и сложные. Но у
данного направления мысли был существенный минус: в настольные игры скучно играть одному. Потому компьютерные игры как
цунами захватило ум школьника. У многих не было денег на компьютер, потому родители покупали своим детям игровые приставки.
Но деньги это еще пол беды. Самое главное, никто не знал зачем нужен компьютер. В восприятии детей это была игрушка, в
восприятии взрослых это была печатная машинка. В то время интернет был только у избранных. Кроме всего прочего этот сложный,
дорогой, вычислительный аппарат нужно было обслуживать. Родители купили персональный компьютер, когда мне было 13 лет.
Счастью моему не было придела. Вот только что дальше? Как сейчас, помню передо мной была клавиатура, с большим количеством
кнопок. Из них я знал, что делают при нажатии, только кнопки с буквами, стрелки и пробел. Заметив, что Word подчеркивает
ошибки в словах я смекнул, как можно быстро делать домашнюю работу по русскому языку. Спустя какое-то время, я нашел
программу переводчик с английского. Ну и конечно взял пару дисков с играми у друга. Вот и весь потенциал который я мог
выжать из компьютера на то время. Это никуда не годилось, было решено отправить меня на курсы пользования компьютером. Я был
этому только рад, ведь я чувствовал, что эта машина способна на большее. Проблема была не в ней, а во мне. Я не знал что
хочу получить от компьютера. И во время обучения на пользователя ПК, где нас учили пользоваться Word, интернетом и другими
базовыми программами, я понял, что я смог ответить на поставленный вопрос. Я хотел заниматься видео-монтажом, рисованием и
созданием компьютерных игр. Я практиковался в видео-монтаже большую часть времени. Я так же пытался пользоваться PhotoShop,
но мышкой толком ничего не нарисуешь, а на обводку одного карандашного рисунка уходило слишком много времени. Насчет создания
компьютерных игр я начал читать статьи в появившемся недавно интернете, заказывать книги. Вскоре, курс пользователя ПК не мог
меня научить ничему новому. Однако я хотел продолжить обучение, но уже в определенной направленности. И из трех направлений,
которыми хотел заниматься, я выбрал программирование. Видео-монтаж я изучил самостоятельно, все что мне оставалось это
оттачивать мастерство. Фотошоп, я вообще невзлюбил из-за мороки с мышкой, и так как на бумаге получалось лучше и быстрее,
благополучно забросил это дело. Оставалось создание компьютерных игр, которое требовало навыка программирования. Что такое
программирование, я даже представления не имел, но для меня это было нечто таким, что могло открыть огромное количество новых
возможностей. На то время мне было 14, а на курсы программирования набирали только с 15. Не помню подробностей, но меня
все-таки приняли на этот курс, и я был самый младший в группе. Это было действительно сложно, но очень интересно. На моих
глазах из ничего зарождался алгоритм, который заставлял машину выполнять только то, что ты от нее требуешь. В процессе
программировании не было ничего лишнего. Только команды, только логика. Сложность нарастала постепенно, печатать приходилось
быстрее, вникать в реализацию таких структур как циклы и массивы, труднее. Чтобы не отставать, от товарищей, я старался
выполнять все домашние задания. Даже, если я не понимал как выполнить задание с использованием той или иной структуры, я
применял, те в которых разбирался. Мы программировали на Turbo Pascal, моей любимой альтернативой for был оператор условий и
go-to. Я считаю, что сложные по своей структуре программы, из простых элементов, развили мое видение расчета операций, поиск
способа достижения решения. Оставалась маленькая деталь --- теоретические знания. К концу учебного года, в навыках
программирования я шел с товарищами на ровне. Все вместе, мы перешли на следующий курс программирования, где нас учили Delphy.
На этом этапе обучения я начал понимать практическую значимость программирования, и уже разрабатывал маленькие приложения под
собственные нужды.

Ответ на вопрос, почему я пошел в IT; я знаю, что я хочу получить от компьютера, а так как компьютер охотно выполняет мои
желания я получаю настоящее удовольствие во время работы за ним.
