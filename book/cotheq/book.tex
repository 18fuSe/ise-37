\documentclass[14pt,a4paper,oneside]{extbook}
\usepackage[utf8]{inputenc}
\usepackage[english,russian]{babel} 
\setcounter{secnumdepth}{0}
\setcounter{tocdepth}{2}
%\usepackage{fontspec} 
%\usepackage[left=0cm, top=0cm]{geometry}
%\defaultfontfeatures{Ligatures={TeX},Renderer=Basic} 
%\setmainfont[Ligatures={TeX,Historic}]{Times New Roman}
\begin{document}
	\begin{titlepage}
		\Huge{\textbf{Тут будет титульник}}
	\end{titlepage}
	\clearpage
	\setcounter{page}{2}

	\tableofcontents
	
	\chapter{Группа 37-1}
		\section{Подгруппа 1}
		
		\subsection{Аяс Боргояков}
		\begin{verbatim}
			http://github.com/Anoque
		\end{verbatim}
		
		\noindent\textbf{Почему я выбрал IT?}\\		
		
		Компьютер появился в 2009 году, до этого <<шарился>> в ноутбуке брата, когда он приезжал из Красноярска. Интернет появлися через полгода после покупки комьпютера.		В то время меня увлекали игры, возможно уже тогда сериалы и документальные фильмы, конечно иногда что-то читал в интернете полезное. Когда-то посмотрел документальный фильм о программистах, хакерах (о них отдельные фильмы). Мне очень понравилось то, что они делали. В принципе я также удивляюсь и каким-либо электронным устройствам. Тогда ещё не думал о том, что хоть как-то буду изучать эту сферу.
		
		Однажды одноклассник попросил меня сделать сайт. Тогда мне было 13 или 14 лет. Конечно я тогда не думал о том, чтобы писать всё самому (даже не представлял как). Самым подходящим вариантом на то момент оказалась CMS под названием DLE (Data life engine). Ну я как и все скачал какой-то шаблон, с очень базовыми знаниями HTML и CSS поправил шаблон под сайт. Функционал конечно не трогал. В целом всё было норм. Только этот школьник оказался ленивым, и ничего не выкладывал на нём. Вот он и загнулся. Даже не помню на каком хостинге покупал домен.
		
		После этого в IT я особо не лез. В школе были базовые задачки на паскале, типа переписать какую-нибудь математическую формулу на паскале и поиграться с входными данными. Иногда лазил по исходникам стандартных проектов, менял какие-то мелочи, конечно особо не понимая что и для чего. Примерно тогда же я достаточно много играл, и делал мини-модификации для игр серии Stalker. У меня был репак, где файлы хранятся не в архиве, а все скрипты находятся просто в своих папках. Я менял эти скрипты и получалась в основном дичь. Но мне это нравилось, и делал мелкие штучки <<по приколу>>. 
		
		Момент, когда начал думать об IT в качестве профессии --- в 9 классе. Тогда был выбор уходить после 9 или учиться до 11. Тогда выбрал в качестве дополнительных экзаменов физику и информатику.
		Планировал идти в 10-11 классы на физмат. Не помню почему, но я решил окончательно поступать после 9 класса куда-нибудь. Единственный для меня вариант был Политехнический колледж на специальности <<Информационные системы>>.
		На тот момент я знал паскаль, чуть чуть PHP, возможно JavaScript (не помню когда начал на нём что-нибудь писать), ну и HTML и CSS. Ну и всякие плюшки в системе Windows 7, из-за чего родственники просили помочь с компом (через это все проходят).
		Ну и в полите сначала учился на 4 и 5. Первый год --- школьная программа. Из IT мало чего изучил. А вот на 2 курсе понеслось. Тогда узнал про базы данных, delphi.
		Совершенно без напряжения всё сдавал на 5. Где-то на 3 курсе вспомнил веб технологии немного, написал первый курсач на делфях с базой данных (данные о студентах, наверно около 7 таблицы с несколькими связями).
		Тут самостоятельно начал изучать ООП в языке Java по видеоурокам и некоторым статьям. Было тяжело понять, не знаю почему у преподавателей не справшивал.
		Пошёл на первую практику в компанию <<Вебтолк>>. Там познакомился с ООП (уже более менее нормально) и такой прекрасной технологией, как менеджер пакетов. В случае с PHP это Composer.
		Там было стандартное задание --- сделать парсер сайта \verb|habrahabr.ru| (тогда он ещё так назывался), и записывать в базу данных последние 10 постов (и перезаписывать, т.е. всегда не больше 10 записей).
		Это делалось с мини-фреймворком Slim, ORM для работы с БД (названия не помню какия именно, начало было таким
		\begin{verbatim}
		ORM::for_table('table')->.......)
		\end{verbatim}
		и парсер (ещё пару дополнений, типа curl, и для вывода объектов в удобном формате).
		За всё это время я ещё ничего клёвого не делал (хоть в какой-то нормальной степени клёвости).
		Вот на 4 курсе мы начали изучать C\# и 1C. C\# мне нравился, 1С не очень. На парах занимался фигней. Что-то верстал (адаптивный дизайн), изучал jQuery.
		Так вот. Можно сказать, что здесь уже реально переломный момент, когда я начал что-то изучать целенаправленно, почти самостоятельно и вроде как более менее получилось.
		Меня направили в Томск на WorldSkills по направлению веб-разработки. Я из всех групп вроде как больше всех шарил в вебе (ага, по сути толком ничего не умел). 
		За 3 дня до отъезда сообщили об этом. Я вообще не был готов. Хоть Ajax в jQuery успел узнать, и некоторые плюшки в Php. Съездил, было клёво, но конечно ничего не занял (хотя всё равно был вне конкурса).
		Зато понял какие задания и какая атмосфера. После приезда начал изучать js, jQuery, php в целом, фреймворк Yii2 и всякие клёвые штуки в CSS (мощная вещь, если уметь пользоваться Для многих вещей и js не нужен, надо быть хитрым и сообразительным :))
		Ну и за пару месяцев уже более менее знал эти языки и фреймворки.
		По Хакасии занял 1 место. Поехал ещё через месяц в Казань. Там толком ничего не сделал (да и не пытался <<затащить>>, т.к. опыта практически и не было, а там чувак всё это уже давно изучил и был реальный опыт).
		К тому же там был косяк с сервером, из-за чего ajax запросы работали криво (точнее чаще всего просто не выполнялись).
		
		Ну и после этого в основном занимаюсь веб-технологиями. Возможно всегда буду заниматься в этой области. Хотя я иногда пробую что-нибудь новое. я уже понимаю, что нельзя останавливаться и лениться.
		Всё постоянно совершенствуется, улучшается, появляются куча всего интересного и мощного. И хочется сделать что-то действительно классное.
		Возможно участвие в опен-сорс проектах, возможно уникальные сервисы, может даже стартап (хотя я ленивый, чтобы что-то самому создавать). 
		Надеюсь, что я буду полезным разработчиком. 
		Нельзя стоять на одном месте, на одной технологии. Конечно мой опыт маленький, разнообразия почти нет, но думаю успею много чего изучить.
		Инодга мне кажется, что нифига не могу. Хотя вроде как и получается. Бывают дни, когда совсем лень что-либо изучать, как будто мозг не хочет принимать информацию, хотя сериальчики --- пожалуйста.
		В некоторые дни на столько прёт, что смотришь на время и удивляешься --- куда это несколько часов так быстро улетели? :D
		Сейчас я стараюсь часто писать всякую фигню и заливать на github. Пробую всякое. Конечно там далеко не всё, т.к. начал заливать что-то около месяца назад. (Можете глянуть, там же видно когда я и что пушил).
		По сути там безделушки. Но стараюсь писать каждый день. Сейчас главное закончить один проект и уже можно заняться ещё чем-нибудь. 
		Человеческий мозг --- странная штука, в один день тебя прёт, в другой ты --- бесполезная биомасса. 
		
		\subsection{Андрей Кот}
		\begin{verbatim}
		http://github.com/cotheq
		\end{verbatim}
		
		\noindent\textbf{Зачем я пошел в IT}\\
		\noindent\textit{Спойлер: 1) потому что мне это было интересно с детства; 2) потому что я уже кое-что знал в этой области}
		
		Всё началось с того, когда у нас появился первый компьютер. Я играл в различные игры (в основном Герои Меча и Магии 3), копался в системных файлах и настройках Windows, рисовал паттерны для рабочего стола... К тому же мне нравилось ходить на работу к отцу, в свободное время мы писали музыку в FL Studio 5. Моё время за компьютером было строго регламентировано (изначально было так, что после того, как я выполню домашнее задание, мне разрешалось не более 30 минут в день), и я всё время тянулся к неизведанному.
		
		Когда мне было около 10 лет, наш домашний компьютер сломался (при включении экран монитора не показывал картинку). И нам пришлось вызывать мастера. Впоследствии оказалось, что слетели настройки BIOS, и надо было всего лишь его обнулить. Тогда я в первый раз заглянул внутрь компьютера. И я стал интересоваться такими темами, как компьютерное железо и установка Windows. В то время у нас ещё не было интернета (он появился только в 2010), поэтому прогресс шёл медленно, а новые знания я получал из разговоров одноклассников и рассматривания характеристик на ценниках в компьютерных магазинах.
		В 2009 году я записался на курсы <<Операционные системы и архитектура ПК>>. Там мы знакомились с компонентами и интерфейсами компьютера, устанавливали различные версии Windows и дистрибутивы Linux, собирали и разбирали компьютеры, собирали конфигурации под определенный бюджет. Помню, как я разобрал компьютер дома, затем собрал его, подсоединив всё кроме USB портов на корпусе. И мне потом приходилось довольно долгое время залезать под стол, чтобы вставить или извлечь какое-нибудь USB-устройство.
		
		Честно говоря, я совсем не помню, когда и откуда я узнал про программирование, но я захотел его изучить. На следующий год я записался на курсы по программированию на языке Pascal ABC. Ничего особенного, мы просто изучали эту среду. Больше всего мне понравилось работать с графикой. Мы рисовали с помощью графических примитивов статичные картинки и анимацию. Тогда я написал две довольно простых игры: космический шутер и арканоид. Мне это очень понравилось, и я захотел продолжать развиваться в этом направлении.
		В 8 и 9 классах школы (2010--2012 годы) моя успеваемость стремительно катилась вниз (у меня по большинству предметов были тройки, реже --- четверки; пятерки были только по информатике и английскому языку), и поэтому я решил, что надо валить из школы. В итоге поступил в колледж на специальность <<Информационные системы>>.
		
		В колледже ничего особенного --- изучил несколько языков программирования, чуть-чуть выучил математику. Защитил диплом по теме, связанной с веб-программированием, и ушел в армию. Спустя год вернулся и захотел вспомнить мой любимый JavaScript. И оказалось, что забыл, как пишется оператор возврата! Но я восстановился буквально за пару дней.
		
		После возвращения из армии у меня было 10 дней, чтобы подать документы. Времени на раздумья совсем не было, а на <<вышку>> поступить всё-таки хотелось. Поэтому я пошел в самый ближайший вуз как в географическом плане (вижу наш корпус из окна), так и по душе. К тому же, выбрал такую специальность, которая стала мне близка за время обучения в школе и колледже. На первом курсе я неоднократно хотел всё бросить и уйти работать, потому что, как мне казалось, не давали никаких новых знаний, но только недавно я действительно осознал, что моё профессиональное развитие зависит только от меня самого.
		
		За 7 лет, что я знаком с программированием, я не стал каким-нибудь очень крутым разработчиком или чемпионом каких-нибудь турниров по программированию. Я вообще не стал гиком в этой области. Из моих достижений в 21 год могу назвать следующие: участие в четвертьфинале ACM в 2017 году, второе место на 11 Открытом турнире по программированию в Абакане. К тому же, какое-то время успел поработать веб-разработчиком.
		
		Мне очень нравится IT в целом и программирование в частности, но я не знаю, действительно ли это то, с чем я бы хотел связать свою жизнь.
		
		\subsection{Иван Гриднев}
		\begin{verbatim}
		http://github.com/GitHubUser228
		\end{verbatim}
		
		\noindent\textbf{Эссе почему я выбрал IT}\\

		В моей семье компьютер был с самого моего рождения. Вообще вся моя семья очень близка к айти, хотя это совсем не их профессианальный пофиль. Мои родители играли в примитивные текствоые онлайн игры, но в отличии от матери отец всегда самостоятельно улучшал эту игру путем написания скриптов, все это видел я, тем самым пасивно подогревая свой интерес к айти сфере. 
		Позже в школе на уроках информатики у нас было очень плохое обучение, наш преподователь просто показывал код на турбопаскале на проекторе и говорил переписывать его 1в1, при этом ничего не объясняя. 
		
		К примеру...
		\begin{verbatim}
			program Hello;
			begin
			writeln ('Hello, world.')
			end.
		\end{verbatim}

		Как раз из-за школы конкретно программирование мне было не интересно, но сама айти сфера была для меня очень близка. Поэтому я поступил в пед колледж на компьютерного техника, а не на программиста. Но тут мое мнение о програмиировании очень сильно изменилось, когда на парах по программированнию нам начали все очень хорошо объяснять, отвечать на наши вопросы и так далее. Можно сказать, что тут во мне радилось желание писать свои собсвенные программы и вообще изучать эту сферу. Все получалось настолько хорошо, что преподователь хвалил меня, хотя я этого точно не ожидал и считал себя очень далеким от непосредственно програмиирования.
		Позже выяснилсось, что на курсовю работу что нам что программистам надо написать базу данных. Для меня это не особо составило труда. Главной проблемой было то, что мой профиль обучения был орентирован на железо, а не на software. Как раз поэтому в колледже я выучил только Delphi и слегка подступился к Arduino. Да, конечно, я изучал сам програмиирование, но на слишком примитивном уровне, к примеру писал макросы для игр и т.п. 
		Позже в армии из-за своей профессии меня отправили сидеть целый год за компьютером. Здесь мои знания пригодилсь только для того, чтобы написать простенький макрос в экселе или отредактировать MySQl БД. Тут я забросил свое обучение, можно назвать это застоем.
		И вот я поступил в университет на програмиирование, потому что лично мне бы хотелось бы изучить айти сферу с другой стороны. За 1.5 года я чувствую как сильно я вырос в плане программирования. Часто ловлю себя на мысли, что если бы я раньше начал учить 
		программирование, то к 22 годам, я бы уже набрался опыта и смог бы лучше использовать свое обучение в университете для того чтобы найти свое место в будущем. Я говорю о участие в олимпиадах и вообще показывал бы себя как программист для того чтобы меня заметили. Я надеюсь, что к концу обучения я сомогу сделать из себя проофессинала и найти то, чем я хочу заниматься и чему хочу посвятить все свое время.\\
		
		\noindent \textbf{Ссылки для того чтобы занять место в эссе}
		\begin{verbatim}
		https://github.com/githubuser228
		https://trello.com/user92149803
		\end{verbatim}
		
		\subsection{Борис Палаш}
		\begin{verbatim}
		http://github.com/lokokoha
		\end{verbatim}
		
		\noindent\textbf{Why did I come in IT?}\\
		
		Мое увлечение персональными компьютерами было довольно длинным и увлекательным путешествием что в конце концов предопределило мою судьбу в бедующей профессиональной деятельности. И так начнем …….
		
		Зимой далекого 2005 года, компьютеризация добралась до и до нашего отдаленного города. Вернувшись с работы отец принес домой наш первый домашний компьютер который был установлен на рабочий стол буквально в считанные минуты. Характеристики данного персонального компьютера повергнут в шок неподготовленного слушателя по этой причине я не буду их озвучивать, скажу только что он был очень большой и громоздкий и имел на борту виндовс 95. Тогда данного аппарата хватило чтоб увлечь меня всего лишь на какие нить пару часов после чего он был успешно забыт и использовался лишь отцом для рабочих манипуляций. Шло время и данный компьютер совершенно распоясался, баги и вирусы буквально выскакивали из него, их можно было потрогать руками и данный компьютер был отправлен в подвал для дальнейшего хранения в течении ближайших 10 лет.
		
		Следующие полгода прошли под эгидой «учиться учиться и еще раз учиться» как завещал великий Ленин и про компьютер не кто не вспоминал но настало лето вместе с ними и долгожданные каникулы. В один из теплых летних деньков моими родственниками было принято решения совершить тактическую вылазку к ближайшему магазину компьютерной техники где за не мыслимые по тем временам 50.000 рублей был выбран и куплен персональный компьютер который радовал нас вплоть до 2014 года, данный компьютер уже умел что то большее чем запускать паинт он уже шел на виндовс хp и спокойно тянул различные игры которые были куплены в том же магазине вместе с кучей внешнего оборудования. Среди разного что было куплено нами была и большая книга в 400 страниц которая рассказывала об устройстве персонального компьютера о том как с ним надо работать, она была проштудирована вдоль и поперек и следом за ней еще пара мелких брошюр на данную тематику. Можно сказать что с данной книги и началась моя любовь к персональному компьютеру. Шло время и в 2010 году нам пришлось переехать в другой город куда вместе с нами и перекочевал компьютер. Так как после переезда мелкому школьнику довольно сложно пройти адаптацию компьютер стал одним из главных моих увлечений на ближайший год благо мою путешествие скрашивал появившийся чуть ранее интернет который позволил мне установить первую и самую мою любимую игру по сей день это World of Warcraft. Вечера проведенные в данной игре были не забываемыми но так как к 10 году я играл в нее уже довольно долго стали появляться сообщения от членов моей гильдии о создании своего пиратского сервера так плавно я и подошел к необходимости изучения программирования и к переломному моменту моей жизни. Свое введение в программирование я начал с ядра Trinity кор когда оно только набирало популярность, изначально все сервера стояли на домашнем компьютере и доступ к ним шел только через популярную в то время программу «Хамач». Но так как большой конкуренции в плане пиратских серверов в то время не наблюдалось спрос даже на такую возможность бесплатно поиграть в заманивал большое количество игроков с разных концов мира. Через небольшой промежуток времени я уже спокойно разбирался в чужом коде и мог писать свой, создание простых сайтов и регистраций далось тоже без больших проблем и тогда настало время одного из первых крупных проектов над которыми мне удалось поработать будучи еще даже школьником.
		
		В один день мне пришло сообщение в скайпе с предложением заняться разработкой и работой на одном из игровых миров пиратского сервера riveras сказать, что я был рад это нечего не сказать, даже не думая я согласился на данную работу так как онлайн данного сервера насчитывал в то время гигантские 5000 + человек. Мне дали возможность поработать на одном из малых риалмов но по прошествии чуть более чем полу года данный сервер был выкуплен более крупным сервером который начинал свой пусть с большого финансового спонсирования это WOWCIRCLE после этого большинство энтузиастов разработчиков было уволено так как они не имели желания работать с мало известным им и не совершеннолетним персоналом.
		
		Таким образом я снова попал в свободное плаванье но уже тогда я понял, что создание игр и разработка контента который приносит и мне и другим людям столько радости это мое. Не описать восторга разработчика когда ты можешь повторить создание своих кумиров а твоих игроков от реализации какого-либо контента стоит еще дороже с этого момента настал период когда языки программирования новые наработки стали поглощаться мной в гигантских объемах. Попав в свободное плаванье я не бросил дело и продолжал реализацию на благо других проектов. В конечном итоге после не долгого скитания я смог найти один из лучших коллективов в пиратских серверах это людей которые решили основать  \verb|https://uwow.biz/| данных сервер стал моим домом на долгие года да и до сих пор я продолжаю заниматься разработкой для него. 
		
		И так заканчивая свой рассказ --- компьютерную биографию я могу сказать, что я нашел то чем я хочу заниматься возможно всю свою жизнь ведь радость от реализации конечного программного продукта сравнима возможно только с прыжком на парашюте (Я не пробывал!) надеюсь хоть один человек дочитал данный текст до победного конца).\\
		
		\noindent\textbf{FULL-STACK DEVELOPER}\\
		Have you ever heard the term <<the whole enchilada?>> Full-stack developers can see projects through from start (prototyping, planning) to middle (designing, building) to finish (deploying, managing). Tasks: Covering the full-stack as a developer means building and managing platforms and working across teams to create projects, code projects, and manage databases. Skills: Anything!! Git + GitHub, HTML, CSS, SASS, JavaScript, Java, MySQL, Ruby, Python, Responsive Web Design, jQuery, frameworks, PHP, Apache\\
		
		Термин full stack описывает разработчика, который одинаково хорошо справляется с написанием фронт-энда и бэк-энда. Чтобы быть более конкретным, это означает, что разработчик может работать с базами данных, PHP, HTML, CSS, JavaScript, а так же может превратить Photoshop-ый макет в готовый к использованию шаблон.
		
		Full stack разработчику не нужно быть супер профессионалом во всех областях и технологиях, в которых он должен работать, потому что это практически невозможно. Он просто должен уметь работать с этими технологиями.\\
		
		\noindent\textbf{Плюсы:}
		\begin{itemize}
			\item Вы можете выбирать, кем работать дальше.\\ Вам гораздо проще сменить ориентацию (простите за двусмысленность), чем обычному разработчику.
			\item Вы видите многое в применении, можете разобраться и понять, что вас интересует.\\ Да, вам придётся потратить время на углубление --- но это будет потраченное с пользой время. Да, вам скорее всего придётся завести несколько пет проджектов, чтобы попробовать всё, что хочется. Но это опять же окупается сторицей
			\item Вы меньше выгораете.\\ Если есть возможность периодически менять проекты, то вы гораздо меньше устаёте от применения одного и того же. Конечно, если вы не хардкорный фанат и не получаете удовольствие просто от того, что пишете всё, скажем, на vanilla C или asm.
			\item Вам проще расти в тимлида или архитектора.\\ Довольно очевидный плюс --- чем больше вы разбираетесь в общей структуре, тем больше у вас шансов на рост в руководителя. Конечно, при наличии желания и коммуникативных навыков.
			\item Вы можете отдебажить всё, что угодно.\\ Очевидный плюс. Ваше системное мышление достигло уровня, на котором вы можете исправить что угодно и где угодно.
			\item Работать веселее, интереснее и познавательнее.\\ За один день вы можете получить много новых навыков и знаний в абсолютно разных вещах.
			\item В одиночку вы можете создавать чудесные вещи на стыке разных технологий.\\ Вы один можете сделать то, на что при стандартном подходе требуется 3-4 человека. Запрограммировать микроконтроллер для интернета вещей, который общается с веб сервером, пишет в базу данных, и данные с которого можно просматривать на веб сайте, в приложении или на мобильном устройстве? Легко! Вы один можете представить всю систему и реализовать её без согласований, недопониманий и проволочек.
			\item Ваши решения работают быстрее и надёжнее.\\ За счёт понимания взаимодействия различных систем, вы можете выбрать лучше пути для их комбинирования. Вы лучше понимаете каждый компонент и не боитесь его использовать. Как пример --- возьмём <<кляудные технологии>> (мопед не мой, в публикациях проскакивало). В общем и целом, облако это чудесный способ решения огромного количества задач, в том числе задач масштабирования. К сожалению, всё чаще вижу, что облачные решения используются просто потому, что разработчик не умеет и боится решить свою задачу как-то ещё, а представляет это в виде дополнительного плюса. А многое можно сделать гораздо дешевле и лучше, если иметь хотя бы поверхностное понимание вопроса.
			\item Вы можете пользоваться почти любыми исходниками.\\ В мире, где решена уже практическая любая прикладная задача, тратить время на то, чтобы написать ещё один велосипед --- просто преступление по отношению к длительности своей жизни. Теперь вы можете взять любой репозиторий на любом языке и воспользоваться им как отправной точкой для своего решения. Вы пролетите свежим бризом над граблями, которые до вас собрали тысячи других разработчиков. Здесь должна была быть картинка, на которой показано, как просветлённый программист в позе Лотоса медленно возносится над горами Тибета, а вокруг него танцуют, обнимаются и водят хоровод языки программирования и различные технологии в образе очаровательных девушек --- но увы, такой картинки почему-то не нашёл. Пожалуйста, представьте это сами.
		\end{itemize}
		
		\subsection{Александр Шаматурин}
		\begin{verbatim}
		http://github.com/ashamaturin
		\end{verbatim}
		
		\noindent\textbf{Эссе <<Почему я пошел в IT?>>}\\
		
		Моё эссе о том, <<почему я пошел в IT?>>. Честно скажу, где-то лет 5 назад, я еще не думал, что я пойду в IT. Я учился в техникуме 4 года, на вентиляционщика. В течении всего обучения в своем колледже, я понял, что на ту специальность которую я учусь, это не совсем мое. То есть, это не в том смысле, что у меня не получалось выполнить какую то работу. А дело в том, что я не чувствовал от этой работы какое-то удовольствие, моя бывшая специальность, является так-же востребовательная, и каждый работодатель нуждается в таком рабочем. Но все равно, как бы наверное это странно не звучало, это не мое, и я не почувствовал ничего, что меня могло бы привлечь туда. А на втором курсе, когда у меня сломался компьютер по не понятным причинам, я решил разобрать и посмотреть в чем там дело, скажу кратко, дело было не системном блоке, а именно проблема с системой. После чего я стал искать много информации в интернете по моей проблеме, и пытался разобраться. Когда нашел все таки подходящую информацию по моей теме, я и приступил к своей работе. И мне пришлось, работать, с командной строкой, и прочими функциями, чтобы после выполнения некоторых операций, компьютер ожил, и я смог дальше восстановить его работу. И после этого момента, я понял, что то что я делал, это было очень интересно, и возможно полезно, я получил от этого удовольствие. Это оказалось, то что мне нужно.
		
		В дальнейшем пока я обучался в техникуме, дополнительно, учил для себя программирование искал информацию о компьютерах, и системах. Искал на чём вообще разрабатывались те или иные программы. И это было очень интересно, потому что программист изучающий языки программирования может:
		
		создавать какую-либо программу
		решать математические, и логичесие задачи
		и т.д
		После не большого ознакомления, решил сам изучить какой-нибудь простой для начинающего человека язык программирования. Мои первым языком стал PascalABC.net, на нем я писал математические примеры, выводя решение на экран. После попробовал создать какую нибудь программу, и по скольку я люблю поиграть иногда в игры, написал игру <<Змейку>> используя обучающее видео по созданием игр на Pascal, и статьи. Не скажу что это очень легко, потому что если сильно не всматриваться в общую структуру разработки игр или решению какой-нибудь задачи, то это с виду кажется очень легко и просто.
		Например, я чут-ли не откаждого третьего человека слышал такие слова:
		
		\textit{<<Да, это все легко, просто взял нарисовал рисунок, создал персонажа(как змейка), написал движение, нарисовал стенку, и нарисовал фрук, чтобы он появлялся в разных местах всегда, и все.>>}
		
		Это с виду снова повторюсь кажется легко, но это не так, надо работать над написанием кодом и изучать. Чтобы например создать графическое окно, и создать в этом окне клетки по которым будет двигаться змейка, создать сам вид змейки, и т.д.
		
		В общем написание программы это не легкое задание. И тут надо трудиться, чтобы все получилось. Скажу честно, мне не дается все это легко и просто, но это занятие мене все равно очень нравится.
		
		И после обучения в техникуме, я сразу подавал документы на поступление в университет.
		
		Мой выбор --- IT-специалист.
		
		IT-специалист --- профессия очень востребовательная и нужная. Такие специалисты нужны в любой компании, независимо от масштаба деятельности, так как в России появляется много заграничных фирм.
		
		Многие привыкли к тому, что выпускники школ выбирают довольно типичные профессии, такие как:
		
		\begin{itemize}
			\item врач
			\item парикмахер
			\item риэлтор
			\item адвокат
			\item учитель
			\item бухгалтер
		\end{itemize}
		
		и другие…
		Более того, многие и не подозревают о существовании более интересных профессий, связанных с различными сферами деятельности. Конечно, все перечисленные мной профессии незаменимы в нашей жизни, но, если человек не плохо разбирается допустим в программировании, почему бы не попытаться попробовать свои силы в профессиях более тяжелых умственных?
		
		Так я и поступил. Мои родители довольны моим выбором профессии. Я считаю, что каждый должен выбирать то, что ему близко, и интересно.
		
		Мне близка работа с компьютером.
		
		Во-первых компьютеры в современном мире стали --- незаменимой вещью, без которой уже трудно жить.
		Во вторых, сейчас не много людей, которые могут работать с компьютером.
		И если очень стараться то можно многого добиться в жизни.
		
		Окончательный ответ на вопрос <<Почему я пошел в IT?>>
		Ко всему написанном, я бы хотел дать окончательный и уже короткий ответ на вопрос <<Почему я пошел в IT?>>, а ответ очень прост. Мне очень нравится заниматься за компьютером, я получаю от этого во-первых удовольствие, во-вторых получаю знания, навыки и умения, которые я могу применять в будущем. И именно по этому я пошел в IT.\\
		
		\noindent Мини цитата\\
		\textit {<<Ну как то так!>>}\\
		
		\subsection{Никита Грузин}
		\begin{verbatim}
		http://github.com/xDololow
		\end{verbatim}
		
		\noindent\textbf{Why did I come in IT?}\\
		
		\Large{Just 'cause}
		
		
		
		
		\section{Подгруппа 2}
	\chapter{Группа 37-2}
		\section{Подгруппа 1}
		\section{Подгруппа 2}

	
\end{document}
