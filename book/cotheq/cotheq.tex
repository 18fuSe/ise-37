\noindent\textit{Спойлер: 1) потому что мне это было интересно с детства; 2) потому что я уже кое-что знал в этой области}
		
		Всё началось с того, когда у нас появился первый компьютер. Я играл в различные игры (в основном Герои Меча и Магии 3), копался в системных файлах и настройках Windows, рисовал паттерны для рабочего стола... К тому же мне нравилось ходить на работу к отцу, в свободное время мы писали музыку в FL Studio 5. Моё время за компьютером было строго регламентировано (изначально было так, что после того, как я выполню домашнее задание, мне разрешалось не более 30 минут в день), и я всё время тянулся к неизведанному.
		
		Когда мне было около 10 лет, наш домашний компьютер сломался (при включении экран монитора не показывал картинку). И нам пришлось вызывать мастера. Впоследствии оказалось, что слетели настройки BIOS, и надо было всего лишь его обнулить. Тогда я в первый раз заглянул внутрь компьютера. И я стал интересоваться такими темами, как компьютерное железо и установка Windows. В то время у нас ещё не было интернета (он появился только в 2010), поэтому прогресс шёл медленно, а новые знания я получал из разговоров одноклассников и рассматривания характеристик на ценниках в компьютерных магазинах.
		В 2009 году я записался на курсы <<Операционные системы и архитектура ПК>>. Там мы знакомились с компонентами и интерфейсами компьютера, устанавливали различные версии Windows и дистрибутивы Linux, собирали и разбирали компьютеры, собирали конфигурации под определенный бюджет. Помню, как я разобрал компьютер дома, затем собрал его, подсоединив всё кроме USB портов на корпусе. И мне потом приходилось довольно долгое время залезать под стол, чтобы вставить или извлечь какое-нибудь USB-устройство.
		
		Честно говоря, я совсем не помню, когда и откуда я узнал про программирование, но я захотел его изучить. На следующий год я записался на курсы по программированию на языке Pascal ABC. Ничего особенного, мы просто изучали эту среду. Больше всего мне понравилось работать с графикой. Мы рисовали с помощью графических примитивов статичные картинки и анимацию. Тогда я написал две довольно простых игры: космический шутер и арканоид. Мне это очень понравилось, и я захотел продолжать развиваться в этом направлении.
		В 8 и 9 классах школы (2010--2012 годы) моя успеваемость стремительно катилась вниз (у меня по большинству предметов были тройки, реже --- четверки; пятерки были только по информатике и английскому языку), и поэтому я решил, что надо валить из школы. В итоге поступил в колледж на специальность <<Информационные системы>>.
		
		В колледже ничего особенного --- изучил несколько языков программирования, чуть-чуть выучил математику. Защитил диплом по теме, связанной с веб-программированием, и ушел в армию. Спустя год вернулся и захотел вспомнить мой любимый JavaScript. И оказалось, что забыл, как пишется оператор возврата! Но я восстановился буквально за пару дней.
		
		После возвращения из армии у меня было 10 дней, чтобы подать документы. Времени на раздумья совсем не было, а на <<вышку>> поступить всё-таки хотелось. Поэтому я пошел в самый ближайший вуз как в географическом плане (вижу наш корпус из окна), так и по душе. К тому же, выбрал такую специальность, которая стала мне близка за время обучения в школе и колледже. На первом курсе я неоднократно хотел всё бросить и уйти работать, потому что, как мне казалось, не давали никаких новых знаний, но только недавно я действительно осознал, что моё профессиональное развитие зависит только от меня самого.
		
		За 7 лет, что я знаком с программированием, я не стал каким-нибудь очень крутым разработчиком или чемпионом каких-нибудь турниров по программированию. Я вообще не стал гиком в этой области. Из моих достижений в 21 год могу назвать следующие: участие в четвертьфинале ACM в 2017 году, второе место на 11 Открытом турнире по программированию в Абакане. К тому же, какое-то время успел поработать веб-разработчиком.
		
		Мне очень нравится IT в целом и программирование в частности, но я не знаю, действительно ли это то, с чем я бы хотел связать свою жизнь.