Компьютер появился у меня где-то в восьмом класс, до этого он был только у мои друзей или в компьютерном клубе где иногда, когда были свободные карманные деньги ходил поиграть в компьютерные игры. Большую часть проводимого времени за компьютером занимали компьютерные игры, просмотр фильмов, скачивания из интернета интересных для себя компьютерного софта.

Момент, когда начал думать о том какую профессию (в какой сфере) выбрать относится к 10 классу. Выбор мой сводился к двум сферам, это биология и связанная с программированием (ИТ). Я выбрал ИТ из-за большой актуальности в данный момент. Из-за того, что информационные технологии все глубже входят в жизнь человека, их влияние можно заметить практически во всех сферах деятельности.

После 11 класса сдав экзамены следи которых была физика, и информатика поступил в Хакасский политехнический колледж на специальности <<Информационные системы>> на 2 курс. В моей школе в предмет информатики не входило изучение Паскаля только основы HTML из-за чего первое время было трудно делать практические задания. Проще стало после начала изучения Delphi. За время обучения в колледже были изучены такие из языков программирования как Pascal, C\#, HTML, 1C, Java script. После четвёртого курса сдав дипломную работу на пять я начал поиски работы, но из-за отсутствия опыта и не увелиности в своих силах и знаниях это у меня плохо получалось. В результате через какое-то время встал вопрос продолжить поиски работы или продолжить обучение. Я решил продолжить обучение и для этого выбрал ХГУ. Я расщипываю что после конца обучения у меня появится достаточно знаний и опыта чтобы в дальнейшем быть квалифицированным специалистом в своей обрасти.    
