В моей семье компьютер был с самого моего рождения. Вообще вся моя семья очень близка к айти, хотя это совсем не их профессианальный пофиль. Мои родители играли в примитивные текствоые онлайн игры, но в отличии от матери отец всегда самостоятельно улучшал эту игру путем написания скриптов, все это видел я, тем самым пасивно подогревая свой интерес к айти сфере. 
		Позже в школе на уроках информатики у нас было очень плохое обучение, наш преподователь просто показывал код на турбопаскале на проекторе и говорил переписывать его 1в1, при этом ничего не объясняя. 
		
		К примеру...
		\begin{verbatim}
			program Hello;
			begin
			writeln ('Hello, world.')
			end.
		\end{verbatim}

		Как раз из-за школы конкретно программирование мне было не интересно, но сама айти сфера была для меня очень близка. Поэтому я поступил в пед колледж на компьютерного техника, а не на программиста. Но тут мое мнение о програмиировании очень сильно изменилось, когда на парах по программированнию нам начали все очень хорошо объяснять, отвечать на наши вопросы и так далее. Можно сказать, что тут во мне радилось желание писать свои собсвенные программы и вообще изучать эту сферу. Все получалось настолько хорошо, что преподователь хвалил меня, хотя я этого точно не ожидал и считал себя очень далеким от непосредственно програмиирования.
		
		Позже выяснилсось, что на курсовю работу что нам что программистам надо написать базу данных. Для меня это не особо составило труда. Главной проблемой было то, что мой профиль обучения был орентирован на железо, а не на software. Как раз поэтому в колледже я выучил только Delphi и слегка подступился к Arduino. Да, конечно, я изучал сам програмиирование, но на слишком примитивном уровне, к примеру писал макросы для игр и т.п. 
		
		Позже в армии из-за своей профессии меня отправили сидеть целый год за компьютером. Здесь мои знания пригодилсь только для того, чтобы написать простенький макрос в экселе или отредактировать MySQl БД. Тут я забросил свое обучение, можно назвать это застоем.
		
		И вот я поступил в университет на програмиирование, потому что лично мне бы хотелось бы изучить айти сферу с другой стороны. За 1.5 года я чувствую как сильно я вырос в плане программирования. Часто ловлю себя на мысли, что если бы я раньше начал учить программирование, то к 22 годам, я бы уже набрался опыта и смог бы лучше использовать свое обучение в университете для того чтобы найти свое место в будущем. Я говорю о участие в олимпиадах и вообще показывал бы себя как программист для того чтобы меня заметили. Я надеюсь, что к концу обучения я сомогу сделать из себя проофессинала и найти то, чем я хочу заниматься и чему хочу посвятить все свое время.
		
		\noindent \textbf{Ссылки для того чтобы занять место в эссе}
		\begin{verbatim}
		https://github.com/githubuser228
		https://trello.com/user92149803
		\end{verbatim}